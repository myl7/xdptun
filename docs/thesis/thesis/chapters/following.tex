% !TeX root = ../main.tex

% Copyright (c) 2022 myl7
% SPDX-License-Identifier: CC-BY-NC-ND-4.0

\chapter{总结与展望}

本项目 xdptun 作为一套 UDP over pseudo-TCP 包伪装方案,能够突破传统 UDP over TCP 方案的局限,不仅仅依赖与 Linux API 提供的网络栈在用户态进行处理,而是借助 eBPF
直接在内核态、于 Linux 内核网络栈外进行包的伪装和解伪装。
这一套崭新的伪装方案以约 10\% 的吞吐量代价,优化了 UDP 流量在公网上的体验,从而为新兴的基于 UDP 的上层网络协议的发展提供了一台支持方案和兼容层,具有一定的实际应用意义。
但这一套方案一方面仍需等待 Linux 新版本内核的推广、等待 eBPF 支持的发展,另一方面也需要针对公网上更细致的协议检测方案做出更多的处理和优化,因而时间上具体实际部署和推广还有一定的距离。

本项目所有设计资源、代码、文档均开源在 GitHub 的 myl7/xdptun 代码仓库中,其中代码部分由于需要和 Linux 内核中 eBPF 相关库链接,在 GPL 2.0 代码许可证的影响下以 GPL 2.0 或者更高版本的许可证发布,具体法律信息可以参见代码随附的 GPL 2.0 许可证文件。

在实现了基本功能的之后,本项目依然存在可以改进和发展的空间:

\section{TCP 碎片伪装为 TCP 连接}

如上述测试与分析章节的功能测试部分指出的,本项目目前伪装出的 TCP 流量呈现为 TCP 碎片。
尽管常见检测方式会将这些 TCP 碎片检测为 TCP 流量,但在通过能处理连接状态的网络流量过滤设备,如带 conntrack 的防火墙时,TCP 碎片会被探测为无效的 TCP 流量而被丢弃。
为了此情况的出现,一个可行的方案是进一步将 TCP 碎片伪装为 TCP 连接。
我们已经设计好了一个完善的解决方案:
为了让这些 TCP 碎片被识别为一个 TCP 连接,在前述设计方案考虑了更改 TCP 头部序列号字段这一操作的前提下,只需通过 eBPF 的持久化手段 maps、将 TCP 碎片的序列号字段排列为连续值即可。
除此之外,还需要伪造开启 TCP 连接的 SYN、ACK-SYN、ACK 三个请求,从而伪装出 TCP 连接创建的过程。
但在此方案下,由于 eBPF 中 maps 持久化方案体现为一个全局的 \texttt{struct} 变量,通过相应的 maps API 进行读写。
而当此程序代码同时运行在多个 CPU 核心上时,同一个 maps 全局变量可能会被多个内核线程同时访问,因而需要加锁或是使用原子操作,进而影响程序的吞吐量。
为了解决这种情况,我们可以使用 maps API 提供的另一套持久化方案,其允许不同内核线程的代码在访问代码中同一个 maps 全局变量时实际访问到不同的数据。
此时,此程序维护与 CPU 核心相同数量的 maps 全局变量和 TCP 连接,不同 TCP 连接使用不同的 TCP 头部序列号序列,从而无需使用锁,维持了程序的吞吐量。

\section{eBPF 生态}

本项目基于 eBPF 实现了核心功能,也使得 eBPF 及其相关生态的问题成为了本项目问题的一部分。
尽管 eBPF 基础功能已经基本完善,但其周边设施依然需要改进和增强,例如 eBPF 相关程序缺少开机启动和后台服务管理的成套设施。
目前 eBPF 周边生态的开发依然主要着眼于实现基础功能的命令行工具,这一方面是 eBPF 新特性不断涌现且随着 Linux 内核版本更新而扩大支持设备范围导致的,另一方面 eBPF 的影响力尚且不足,在缺少大公司支持、依靠开源生态推广的情况下难以吸引到足够的周边设施开发者。
如果需要进一步推广本项目,甚至投入到实用中,周边工具的开发还有待精力投入。
