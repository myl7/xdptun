% !TeX root = ../main.tex

% Copyright (c) 2022 myl7
% SPDX-License-Identifier: CC-BY-NC-ND-4.0

\chapter{缺陷与改进}

\section{TCP 碎片伪装为 TCP 连接}

如上述测试与分析章节的功能测试部分指出的,本项目目前伪装出的 TCP 流量呈现为 TCP 碎片。
尽管常见检测方式会将这些 TCP 碎片检测为 TCP 流量,但在通过能处理连接状态的网络流量过滤设备,如带 conntrack 的防火墙时,TCP 碎片会被探测为无效的 TCP 流量而被丢弃。
为了此情况的出现,一个可行的方案是进一步将 TCP 碎片伪装为 TCP 连接。
我们已经设计好了一个完善的解决方案:
为了让这些 TCP 碎片被识别为一个 TCP 连接,在前述设计方案考虑了更改 TCP header 序列号字段这一操作的前提下,只需通过 eBPF 的持久化手段 maps、将 TCP 碎片的序列号字段排列为连续值即可。
除此之外,还需要伪造开启 TCP 连接的 SYN、ACK-SYN、ACK 三个请求,从而伪装出 TCP 连接创建的过程。
但在此方案下,由于 eBPF 中 maps 持久化方案体现为一个全局的 \texttt{struct} 变量,通过相应的 maps API 进行读写。
而当此程序代码同时运行在多个 CPU 核心上时,同一个 maps 全局变量可能会被多个内核线程同时访问,因而需要加锁或是使用原子操作,进而影响程序的吞吐量。
为了解决这种情况,我们可以使用 maps API 提供的另一套持久化方案,其允许不同内核线程的代码在访问代码中同一个 maps 全局变量时实际访问到不同的数据。
此时,此程序维护与 CPU 核心相同数量的 maps 全局变量和 TCP 连接,不同 TCP 连接使用不同的 TCP header 序列号序列,从而无需使用锁,维持了程序的吞吐量。

\section{eBPF 生态}

本项目基于 eBPF 实现了核心功能,也使得 eBPF 及其相关生态的问题成为了本项目问题的一部分。
尽管 eBPF 基础功能已经基本完善,但其周边设施依然需要改进和增强,例如 eBPF 相关程序缺少开机启动和后台服务管理的成套设施。
目前 eBPF 周边生态的开发依然主要着眼于实现基础功能的命令行工具,这一方面是 eBPF 新特性不断涌现且随着 Linux 内核版本更新而扩大支持设备范围导致的,另一方面 eBPF 的影响力尚且不足,在缺少大公司支持、依靠开源生态推广的情况下难以吸引到足够的周边设施开发者。
如果需要进一步推广本项目,甚至投入到实用中,周边工具的开发还有待精力投入。
