% !TeX root = ../main.tex

% Copyright (c) 2022 myl7
% SPDX-License-Identifier: CC-BY-NC-ND-4.0

\ustcsetup{
  keywords = {
    计算机网络,增强的伯克利包过滤器,传输控制协议上的用户数据报协议
  },
  keywords* = {
    Computer networking, eBPF, UDP over TCP
  },
}

\begin{abstract}
  由于 UDP 协议功能和承诺较少,且实际使用中常常承载娱乐性的流量、对服务质量不敏感,以往 UDP 在 QoS 中收到不公平对待,UDP 流量亦在公网中质量不佳。
  但随着新一代基于 UDP 的上层网络协议的出现,UDP 流量的重要性有了较大的提升,因而需要处理方案从而在当前网络环境下保证 UDP 流量的质量。
  常见方案是各种 UDP over TCP 方案,但是这些方案往往实现简单而性能不佳,无法提供透明的转换,也无法轻松复用现有的网络工具。
  而随着 eBPF 技术逐渐步入成熟,eBPF 中对于网络数据包处理的支持 XDP 和 TC BPF 提供了一套全新的方案以实现高效的包处理。
  XDP 和 TC BPF 允许程序在 Linux 内核态中、Linux 内核网络栈外捕捉到原始数据帧,并允许 eBPF 程序对其进行修改以改变帧中数据,这使得在 eBPF 之中实现一套 UDP over TCP 机制成为可能。

  本文即提供了一套基于 eBPF 的 UDP 包伪装方案 xdptun,克服了 eBPF 高度受限的开发环境,通过在 Linux 内核网络栈入口最前端的 XDP 处和出口最后端的 TC BPF 处加载两份 eBPF 程序进行网络数据帧处理、将 UDP 头部与 TCP 头部互相转换并在帧尾留存从头部末尾处移出的数据,从而将 UDP 流量高效地、透明地伪装为 TCP 流量。
  经过测试,本地环境下与无其他处理相比,部署此方案时性能上仅有约 10\% 的吞吐量损失。
  测试中在部署后 UDP 数据包被转化为类似 TCP 碎片的帧格式,将被简单的协议检测机制检测为 TCP,并可于接收端被转换会 UDP 数据包。

  尽管此方案尚且无法直接工作在公网上,但基本功能已经完整,相应扩展也已留有方案和空间。
  此方案能够为基于 UDP 的上层网络协议提供一份当前网络环境下的兼容层,保障这些新一代网络协议在当下的正常高效运行。
\end{abstract}

\begin{abstract*}
  Considering functions and promises included in UDP are little, and in the real network environment, UDP usually ships entertainment traffic which cares little about traffic quality, UDP traffic used to receive unfair treatment in QoS and be inefficent in the Internet.
  But as the new generation of upper layer network protocol comes out, the importance of UDP traffic increases, and we required solutions to ensure the quality of UDP traffic in current Internet.
  Common solutions are various UDP over TCP, but current implementations are too simgle so that they are not efficient, can not provide transparent processing, and are difficult to reuse existing network stack.
  And with the development and maturity of eBPF, network packet processing support of eBPF, XDP and TC BPF, provide a new solution to implement a high efficiency packet processing.
  XDP and TC BPF allow programs to be executed in Linux kernel mode but kept out of Linux kernel network stack, to capture raw data frames and edit these frames, which makes it possible to do implement UDP over TCP in eBPF.

  This paper provides a solution based on eBPF to do UDP packet obfuscation named xdptun, which overcomes the highly limited development environment of eBPF, loads two eBPF programs in the front of Linux kernel network stack, XDP, and in the tail of Linux network stack, TC BPF, to do network data frame processing, updates UDP header to or from TCP header, and leaves moved data to the end of the frame which is extended, in order to obfuscate UDP traffic to TCP traffic efficiently and transparently.
  According to tests, compared with the case that does not configured xdptun, xdptun only causes about 10\% throughpput fall in the locally connected network.
  In tests, UDP packets are transformed into frames like TCP segments, which will be recognized as TCP traffic by simple protocol detection mechanism, and can be received and transformed back in the receiving end.

  Though so far xdptun can still not work practically on the Internet, the basic function is complete and the extension is considered and prepared.
  The solution can provide the upper layer network protocols which are based on UDP with a compatible layer in current network environment to ensure these protocols can work correctly and efficiently.
\end{abstract*}
