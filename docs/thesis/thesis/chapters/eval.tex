% !TeX root = ../main.tex

% Copyright (c) 2022 myl7
% SPDX-License-Identifier: CC-BY-NC-ND-4.0

\chapter{测试与分析}

\section{功能测试}

\subsection{设备配置}

为了测试此应用在公网的复杂环境中的实际表现,我们选取了两台拥有公网 IP、处于不同国家间的服务器进行测试。
为了便于分辨,我们将其中一台设备记为 A1 端,将另一台设备记为 A2 端。
A1 A2 两端均为 KVM 虚拟机,操作系统均为 Ubuntu 20.04,内存均为 2GB,网口最大带宽均为共享型的 1000Mbps。
其中 A1 位于中国香港,A2 位于美国西海岸,其间网络通过太平洋跨洋电缆互相直接连接,不进行跨国中继跳转。

一个需要提到的状况是,由于这两台机器作为 VPS(Virtual Private Server)均为 KVM(Kernel-based Virtual Machine)虚拟机,入方向程序挂载在之上时,受虚拟机中虚拟网卡设备对应的驱动的限制,XDP 均工作在 SKB 模式下。
但由于我们在此次实验中仅测试面对公网环境时的功能情况,因此 XDP 挂载模式产生的性能问题不会影响实验结果。

\subsection{测试流程}

与性能测试中 WireGuard over xdptun 的测试样例相似,我们同样在 A1 端和 A2 端上架设好了 WireGuard 和 xdptun,然后使用 Python 3 标准库中 http 模块的 server 子模块 来启动一个 HTTP 服务器进行文件下载测试,观察下载是否能够成功。
此处考虑到测试结果而暂时略过具体方案的描述,而是在后续的性能测试中给出。

\subsection{测试结果}

测试结果显示,作为发送端的 A1 端无法将文件发送至接受端 A2 端。
经网络接口处的 tcpdump 抓包后、再使用网络抓包工具 WireShark 进行分析后,我们首先确认了 xdptun 的功能实现没有问题,在 WireShark 中数据包被简单的流量类型识别判定为了 TCP 流量,不过由于缺少建立 TCP 连接的过程,这些分散的数据包被认为是 TCP 碎片并被示警。
进一步我们推定了出现上述状况的原因是公网上 A1 端发送给 A2 端的 TCP 碎片被携带 conntrack 功能的网络流量过滤工具如防火墙等丢弃,从而导致了 A1 端数据无法发往 A2 端。
对于此问题我们还会在之后章节内进行更详细的分析并给出解决方案。

\section{性能测试}

\subsection{设备配置}

为了测试本项目的性能,我们使用了一台笔记本设备和一台树莓派设备来进行测试。
为便于分辨,我们将笔记本端记为 B1 端,将树莓派端记为 B2 端。
B1 端作为笔记本,设备型号为 Yoga Slim 7 Pro-14ACH5 Laptop (ideapad) - Type 82MS,内存为 16GB,网卡支持最大带宽为 1000Mbps,操作系统为 Arch Linux 从而方便使用较新的 Linux 内核,Linux 内核版本为 5.16.14-arch1-1。
B2 端作为树莓派设备,设备型号为 Raspberry Pi 3 Model B+,内存为 2GB,网卡支持最大带宽为 300Mbps,实际经由 USB 2.0 接口转换获得所以无法达到此最大速率,操作系统为 Raspberry Pi OS (64-bit) 而此系统镜像是基于 Debian bullseye 即 Debian 11 修改而来,Linux 内核版本为 5.15.28-v8+ 并由我们进行了一些小修改,此修改将在后续部分介绍。
特别地,由于 B2 端树莓派设备是嵌入式设备,供电可能会影响 CPU 性能进而影响试验结果,特此指出 B2 端由 B1 端的 USB 接口进行供电。
B1 B2 两端间经由 RJ45 接口的网线连接。

\subsection{Raspberry Pi OS XDP 特殊处理}

由于树莓派设备上的 Raspberry Pi OS 系统不支持 TC BPF,且 XDP 支持也存在缺陷、需要额外的修改才能正确完成收缩数据帧长度的系统调用。
所以在实验前,我们首先重新编译 Raspberry Pi OS 的 Linux 内核,对一部分代码进行了修改以启用此功能。
具体的原因分析及修改方案如下:

Raspberry Pi OS 中 XDP 出现期望外现象的位置是 Linux API \texttt{bpf\_xdp\_adjust\_tail} 函数。
在此函数中,内核将检查当前网卡驱动支持的最大帧长,从而保证在使用 \texttt{bpf\_xdp\_adjust\_tail} 函数扩展帧长时不会超过最大帧长。
这需要网卡驱动设置了对应的最大帧长,否则内核将会将最大帧长初始化为一个极大值。
而 Raspberry Pi OS 中对应的网卡驱动即没有设置此值,导致上述检查失败,\texttt{bpf\_xdp\_adjust\_tail} 函数提前返回。
但由于本项目中仅使用 \texttt{bpf\_xdp\_adjust\_tail} 函数收缩帧长、不会导致帧过长的问题,所以我们可以安全地将此检查移除,从而使得测试能够在 Raspberry Pi OS 上正常进行。

\subsection{测试流程}

为了进行测试,在作为发送端的 B1 端上,我们首先用随机数据生成一个 1 GiB(1073741824 字节)大小的测试用大文件。
然后,借助 Python 3 标准库中 http 模块的 server 子模块,我们启动一个 HTTP 服务器以提供测试用文件的下载。
在搭建好测试用的服务后,通过架设不同的环境于同一组设备 B1 端和 B2 端上,就可以测试使用 xdptun 与否的情况下性能的差异。
这里我们安排了三套环境进行测试:

\begin{itemize}
  \item B1 端与 B2 端完全不变,进行从 B1 端到 B2 端的文件下载测试,作为空白对照组;
  \item B1 端与 B2 端间架设好 WireGuard 并让 UDP 协议的 WireGuard 流量工作在两端间,进行下载测试,作为对照组;
  \item B1 端与 B 间除了架设好 WireGuard 外,还架设好 xdptun 以让 UDP 协议的 WireGuard 流量经由 xdptun 的两端转化为 pseudo-TCP 流量进行传输,作为试验组。
\end{itemize}

由于实验中 B1 端与 B2 端进行了本地的有线连接,所以无法进一步测试丢包等情况下的状况,但是可以最大限度地测试 xdptun 在吞吐量上的性能影响。
测试中下载大文件作为测试的过程使用 cURL 这一经典网络客户端实现作为工具。
额外的,在经过实际测试确认 B1 端与 B2 端连通后,还需要注意移除 \texttt{bpf\_printk} 函数。
根据 Linux 文档介绍,涉及 IO 的 \texttt{bpf\_printk} 是一个可能的性能损失点,移除它们才能避免不必要操作对测试中软件性能的干扰。
环境配置完成后,针对每个环境,我们测试三遍同一大文件下载时间并观察 CPU 和内存等的占用情况,然后取平均值作为结果。

\subsection{测试结果}

我们在测试中获得的数据如表~\ref{tab:perf-test}所示。

\begin{table}[h]
  \centering
  \caption{性能测试结果}
  \label{tab:perf-test}
  \begin{tabular}{c c c c c}
    \toprule
    测试环境 & 第一次结果 & 第二次结果 & 第三次结果 & 平均值 \\
    \midrule
    HTTP & 15.1 & 14.8 & 15.0 & 15.0 \\
    HTTP over WireGuard & 10.8 & 11.0 & 10.9 & 10.9 \\
    HTTP over WireGuard over xdptun & 9.80 & 9.80 & 9.80 & 9.80 \\
    \bottomrule
  \end{tabular}
  \note{结果数字均为完成下载所需的时间,单位为秒}
\end{table}

在实验过程中,我们还可以观察到,无论是三种中哪一种情况,作为接收端的 B2 端的 CPU 都维持在一个很高的水平。
但即便是这样,通过分析表中数据,我们也可以发现,xdptun 仅会对带宽造成约 10\% 的损失,是一个可以接受的数值。
