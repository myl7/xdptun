% !TeX root = ../main.tex

% Copyright (c) 2022 myl7
% SPDX-License-Identifier: CC-BY-NC-ND-4.0

\chapter{测试与分析}

\section{性能测试}

\subsection{设备配置}

为了测试本项目的性能,我们使用了一台笔记本设备和一台树莓派设备来进行测试。
为便于分辨,我们将笔记本端记为 A 端,将树莓派端记为 B 端。
A 端作为笔记本,设备型号为 Yoga Slim 7 Pro-14ACH5 Laptop (ideapad) - Type 82MS,网卡支持最大带宽为 1000Mbps,操作系统为 Arch Linux 从而方便使用较新的 Linux 内核,Linux 内核版本为 5.16.14-arch1-1。
B 端作为树莓派设备,设备型号为 Raspberry Pi 3 Model B+,网卡支持最大带宽为 300Mbps,实际经由 USB 2.0 接口转换获得所以无法达到此最大速率,操作系统为 Raspberry Pi OS (64-bit) 而此系统镜像是基于 Debian bullseye 即 Debian 11 修改而来,Linux 内核版本为 5.15.28-v8+ 并由我们进行了一些小修改,此修改将在后续部分介绍。
特别地,由于 B 端树莓派设备是嵌入式设备,供电可能会影响 CPU 性能进而影响试验结果,特此指出 B 端由 A 端的 USB 接口进行供电。
A B 两端间经由 RJ45 接口的网线连接。

\subsection{Raspberry Pi OS XDP 特殊处理}

由于树莓派设备上的 Raspberry Pi OS 系统不支持 TC BPF,且 XDP 支持也存在缺陷、需要额外的修改才能正确完成收缩数据帧长度的系统调用。
所以在实验前,我们首先重新编译 Raspberry Pi OS 的 Linux 内核,对一部分代码进行了修改以启用此功能。
具体的原因分析及修改方案如下:

Raspberry Pi OS 中 XDP 出现期望外现象的位置是 Linux API \texttt{bpf\_xdp\_adjust\_tail} 函数。
在此函数中,内核将检查当前网卡驱动支持的最大帧长,从而保证在使用 \texttt{bpf\_xdp\_adjust\_tail} 函数扩展帧长时不会超过最大帧长。
这需要网卡驱动设置了对应的最大帧长,否则内核将会将最大帧长初始化为一个极大值。
而 Raspberry Pi OS 中对应的网卡驱动即没有设置此值,导致上述检查失败,\texttt{bpf\_xdp\_adjust\_tail} 函数提前返回。
但由于本项目中仅使用 \texttt{bpf\_xdp\_adjust\_tail} 函数收缩帧长、不会导致帧过长的问题,所以我们可以安全地将此检查移除,从而使得测试能够在 Raspberry Pi OS 上正常进行。

\subsection{测试流程}

为了进行测试,在作为发送端的 A 端上,我们首先用随机数据生成一个 1 GiB(1073741824 字节)大小的测试用大文件。
然后,借助 Python 3 标准库中 http 模块的 server 子模块,我们启动一个 HTTP 服务器以提供测试用文件的下载。
此 HTTP 服务器受实现的限制,仅提供 HTTP/1.0 协议的访问,但考虑到所有 HTTP 协议中最常见的 HTTP/1.1 协议与 HTTP/1.0 协议差距不大,我们有理由认为此实验可以模拟真实网络环境中的 HTTP 访问。
在搭建好测试用的服务后,通过架设不同的环境于同一组设备 A 端和 B 端上,就可以测试使用 xdptun 与否的情况下性能的差异。
这里我们安排了三套环境进行测试:

\begin{itemize}
  \item A 端与 B 端完全不变,进行从 A 端到 B 端的文件下载测试,作为空白对照组;
  \item A 端与 B 端间架设好 WireGuard 并让 UDP 协议的 WireGuard 流量工作在两端间,进行下载测试,作为对照组;
  \item A 端与 B 间除了架设好 WireGuard 外,还架设好 xdptun 以让 UDP 协议的 WireGuard 流量经由 xdptun 的两端转化为 pseudo-TCP 流量进行传输,作为试验组。
\end{itemize}

由于实验中 A 端与 B 端进行了本地的有线连接,所以无法进一步测试丢包等情况下的状况,但是可以最大限度地测试 xdptun 在吞吐量上的性能影响。
测试中下载大文件作为测试的过程使用 cURL 这一经典网络客户端实现作为工具。
额外的,在经过实际测试确认 A 端与 B 端连通后,还需要注意移除 \texttt{bpf\_printk} 函数。
根据 Linux 文档介绍,涉及 IO 的 \texttt{bpf\_printk} 是一个可能的性能损失点,移除它们才能避免不必要操作对测试中软件性能的干扰。
环境配置完成后,针对每个环境,我们测试三遍同一大文件下载时间并观察 CPU 和内存等的占用情况,然后取平均值作为结果。

\subsection{测试结果}

我们在测试中获得的数据如表~\ref{tab:perf-test}所示。

\begin{table}[h]
  \centering
  \caption{性能测试结果}
  \label{tab:perf-test}
  \begin{tabular}{c c c c c}
    \toprule
    测试环境 & 第一次结果 & 第二次结果 & 第三次结果 & 平均值 \\
    \midrule
    HTTP & 15.1 & 14.8 & 15.0 & 15.0 \\
    HTTP over WireGuard & 10.8 & 11.0 & 10.9 & 10.9 \\
    HTTP over WireGuard over xdptun & 9.80 & 9.80 & 9.80 & 9.80 \\
    \bottomrule
  \end{tabular}
  \note{结果数字均为完成下载所需的时间,单位为秒}
\end{table}

在实验过程中,我们还可以观察到,无论是三种中哪一种情况,作为接收端的 B 端的 CPU 都维持在一个很高的水平。
但即便是这样,通过分析表中数据,我们也可以发现,xdptun 仅会对带宽造成约 10\% 的损失,是一个可以接受的数值。

\section{功能测试}

\subsection{设备配置}

进一步地,为了测试此应用在公网的复杂环境中的实际表现,我们也选取了两台拥有公网 IP、处于不同国家间的服务器进行测试,具体相关参数如表<TOREF>所示。
同样是为便于分辨,我们将其中位于中国香港处的设备记为 C 端,将位于美国西海岸处的设备记为 D 端。
需要提到的是,由于这两台机器作为 VPS 均为 KVM 虚拟机,入方向程序挂载在之上时,受虚拟机中虚拟网卡设备对应的驱动的限制,XDP 均工作在 SKB 模式下。
但由于我们在此次实验中仅测试面对公网环境时的功能情况,因此 XDP 挂载模式产生的性能问题不会影响实验结果。

\subsection{测试流程}

与性能测试中 WireGuard over xdptun 的测试样例相似,我们同样在 C 端和 D 端上架设好了 WireGuard 和 xdptun,然后使用 Python 3 标准库中 http 模块的 server 子模块 来启动一个 HTTP 服务器进行文件下载测试,观察下载是否能够成功。

\subsection{测试结果}

测试结果显示,作为发送端的 C 端无法将文件发送至接受端 D 端。
经网络接口处的 tcpdump 抓包后、再使用 WireShark 进行分析后,我们首先确认了 xdptun 的功能实现没有问题,因而进一步断定了出现上述状况的原因是公网上 C 端发送给 D 端的 TCP 碎片被携带 conntrack 功能的网络流量过滤工具如防火墙等丢弃,从而导致了 C 端数据无法发往 D 端。
对于此问题我们还会在之后章节内进行更详细的分析和处理。
