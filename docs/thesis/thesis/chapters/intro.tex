% !TeX root = ../main.tex

% Copyright (c) 2022 myl7
% SPDX-License-Identifier: CC-BY-NC-ND-4.0

\chapter{绪论}

\section{传输层协议倾向变化}

过去传输层上,在两类主流的传输层协议 TCP 和 UDP 中,相较于 UDP,TCP 保持者绝对的主要地位,专家设计上层协议时也往往在传输层选择 TCP 来承载流量,甚至是直接选择基于 TCP 的 HTTP 作为基础。
这是因为 UDP 存在一些对于当时的服务而言影响较大的几个缺点:
首先 UDP 协议不承诺送达和去重\cite{rfc768}、不保证稳定传输,在上层协议需要实现稳定传输时会加重上层协议设计的负担;
其次亦因为 UDP 不承诺送达,在公网 QoS 中 QoS 执行者常常会降低 UDP 流量权重从而保证主要的服务承载流量 TCP 流量的质量,甚至是在资源不足、硬件性能受限或是存在诸如 DDoS<TOREF UDP 反射攻击> 等安全问题的情况下直接阻断 UDP 流量,进而造成了 UDP 流量时常质量不佳的情况。
此情况尤其常见于廉价家庭宽带接入、跨运营商通信和跨国通信等环境,对普通用户服务可达性、公司服务群及内网建设等有较大的负面影响。

但在最近几年的网络协议发展中,TCP 一家独大的趋势已经逐渐开始发生变化,而 UDP 也因为其简单灵活的特点而越来越多地被选用。
一个典型的例子是 QUIC 及 HTTP/3 的出现。
HTTP 是互联网应用层中无可匹敌的主流协议。
在过去基于 TCP 的 HTTP 协议中,HTTP/1.1 和 HTTP/2 是主流选择。
其中 HTTP/1.1 由于每次请求都会打开一个新连接,资源消耗大且交互延迟高;
而 HTTP/2 则可以复用 TCP 连接,部分解决了新连接开销大的问题,但也存在队头阻塞的问题。
队头阻塞是指 TCP 拥塞控制中多个 HTTP 连接复用同一个 TCP 连接时,假如 TCP 消息队列队头的包发生了超时丢包,则此丢包会阻塞后续所有的 HTTP 连接,即使这些 HTTP 连接之间没有依赖关系。<TOREF 队头阻塞>
尽管有 BBR\cite{45646} <TOREF BBR2>等项目致力于在 TCP 发生超时丢包时优化拥塞控制策略、尽快恢复原传输速率,但仍然无法彻底解决这个问题。
而 HTTP/3 的提出便是为了解决这个问题。
HTTP/3 由 Google 主导开发,其最初是在 UDP 协议上实现了能够进行稳定传输的 QUIC 协议,后来在 2018 年 Google 提议将 HTTP over QUIC 重命名为 HTTP/3<TOREF> 以作为新一代 HTTP 标准,并于同年获得了 IETF 成员的认可<TOREF>。
尽管截至目前 HTTP/3 仍是草案状态\cite{ietf-quic-http-34},但浏览器 Google Chrome 和 Firefox、CDN 厂商 Cloudflare 等均已部署了对 HTTP/3 的支持。
由于 HTTP/3 基于 QUIC 而 QUIC 基于 UDP,在 HTTP/3 这样的新一代上层协议中,UDP 的重要性出现了显著的上升,UDP 流量质量不佳问题的影响面也开始逐渐扩大。

不仅有 HTTP/3,WireGuard 也是一个例子。
WireGuard 是 Linux 5.6 被合并入 Linux 内核<TOREF>的、新一代 VPN 协议。
WireGuard 具有全平台、高性能、易配置、支持后量子安全<TOREF>等优点,能够极大地方便公司内网建设、自建 AS 等组网问题。
WireGuard 选择了 UDP 协议来进行节点间数据传输,以规避了 TCP over TCP 时的性能问题,并允许复用现有的 UDP over TCP 方案来降低其实现的复杂度。
此选择同样增强了 UDP 的重要性,推动了对于解决 UDP 流量质量不佳问题方法的探索。

\section{UDP over TCP 的针对优化}

为了解决 UDP 流量质量不佳的问题,目前的主流方案是利用 TCP 流量在常见 QoS 策略中的优越性,将 UDP 流量伪装为 TCP 流量,亦即 UDP over TCP。
具体实现有 GOST<TOREF gost repo> 等项目。
也有一些其他方案,例如:UDP over HTTP,从而不必自行管理 TCP 连接;UDP over WebSocket,从而提高实时性且方便兼容使用非 TCP 的流式协议作为下层协议的网络栈。

但这些 UDP over TCP 或者是其他 over 更上层协议的方案存在一些关键问题,例如:在调用操作系统接口时,协议的封装和解封装是完整通过 Linux 内核网络栈后在用户态进行的,一方面途经 Linux 内核网络栈的过程和通用 Linux API 的限制使得应用存在性能优化空间,另一方面用户态获得的原始包无法直接复用 Linux 内核网络栈的各个工具(e.g. TC、Netfilter、IP Set 等)。这些问题的出现实际上是因为目前的主流 UDP over TCP 实现仅仅是简单地将 UDP 流量在 TCP 中传输,没有针对 UDP 自由发送而 TCP 进行流式传输的差异进行特定场景的针对性优化。
而本项目 xdptun 即进行了在 UDP 与 TCP 互相转换时的针对性优化,尽管依然在协议上选择了 UDP over TCP,但是借助 Linux 可编程内核机制 eBPF,直接提前接管 L2-L4 的协议处理,从而一方面不必实现完整的下层 TCP 协议、仅需模拟 UDP 流量为 TCP 流量实现伪装,另一方面完成封装前或解封装后可以复用 Linux 内核网络栈工具,方便进一步处理并避免重复工作。

\chapter{理论基础}

\section{eBPF}

eBPF 是 Linux 内核提供的一套编程机制。
它允许开发者安全地扩展 Linux 内核功能而不需要修改内核代码、不需要重新编译内核。
eBPF 以一个二进制文件的形式被加载到内核中,首先经由 BPF verifier 校验安全性,校验通过后被解释执行。
其 runtime 带有 JIT,并允许通过 maps 进行数据持久化并与用户态程序沟通<TOREF>。

早期 BPF 作为网络包处理机制,架构较简单,功能非常受限。但从 2011 年开始 BPF 机制受到了 Linux 内核开发者的进一步开发,从而使得其功能有了较大的增强。
改进增强后的 BPF 则称为 eBPF(i.e. extended BPF)或是直接沿用 BPF 这个名称,而曾经的 BPF 则称作 cBPF(i.e. classic BPF)。\cite{10.1145/3371038}
本文中后续所有的 BPF,包括出现在特定术语中的 BPF,如无特殊解释,均指 eBPF。

为了保证外来代码的安全性、防止外来代码导致 Linux 内核崩溃,作为内核编程机制的 eBPF 强制要求一套代码安全校验机制。
首先是 eBPF 代码环境,尽管是使用 C 语言进行编程,但程序无法访问 C 语言标准库(或是更明确地说,无法访问 libc),只能调用 Linux 内核提供的一些 API 完成一些功能。
这也允许了编写的代码被编译到一个的特殊 BPF 编译目标、包含 eBPF 指令到二进制文件中从而允许 BPF 解释器执行这些 eBPF 代码。
进一步的,编译好的 eBPF 二进制文件还需要通过 BPF verifier 的校验。
BPF verifier 是 Linux 内核携带的一个 eBPF 校验器,可以使用形式化验证的方法校验 eBPF 文件是否会导致非法的内存访问,从而完美地保证此 eBPF 代码不会使得 Linux 内核崩溃。
BPF verifier 进一步地增大了用 C 语言编写 eBPF 程序时的难度,例如 C 语言中仅能从 API 提供的安全指针开始、在校验好边界的范围内生成新指针、而不允许从不安全的指针获得新指针后再进行校验。
这些最终使得 eBPF 程序编写难度很高且需要熟悉 BPF verifier 的校验机制,而这也是本项目的难点之一。
在后续的设计方案章节中将会有众多不易理解、看似冗余的代码操作,便是为了通过 BPF verifier 而做的。

\section{XDP 和 TC BPF}

XDP 和 TC BPF 是 eBPF 在网络包处理方面的两个接口。
XDP 仅工作在入方向,位于整个 Linux 内核网络栈最前端、进行于 SKB 分配之前,对每一将要进入 Linux 内核网络栈的数据帧进行处理。
挂载在 XDP 上的 eBPF 程序可以直接访问接受到的数据帧的原始数据,并可通过返回值控制此数据帧的行为、进行包过滤、修改甚至是以极低的代价避开 Linux 内核而直接将数据帧发往用户态程序。
同时由于 XDP 工作于 Linux 内核为数据包进行内存分配之前,其包处理性能非常优异,吞吐量大、延迟低。
而 TC BPF 而可以工作在出入两方向,进行于 SKB 分配之后。
由于 TC BPF 的出方向位于整个 Linux 内核网络栈最末端,在 TC BPF 出方向中可以透明地对数据包进行修改,从而避免影响其他 Linux 工具的可复用性。
本项目中我们选择了 XDP 和 TC BPF 出方向这两个接口进行开发,从而在 eBPF 中实现透明的 UDP over TCP。
