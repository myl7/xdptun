% !TeX root = ../main.tex

% Copyright (c) 2022 myl7
% SPDX-License-Identifier: CC-BY-NC-ND-4.0

\chapter{简介}

\section{协议背景}

\subsection{UDP 和 TCP}

由于 UDP 协议不承诺送达和去重\cite{rfc768}、不保证稳定传输,公网 QoS 中常常会降低 UDP 流量权重而保证主要服务承载流量 TCP 流量的质量,甚至是在资源不足、硬件性能受限或是存在安全问题的情况下直接阻断 UDP 流量,进而造成了 UDP 流量时常质量不佳的情况。此情况尤其常见于廉价家庭宽带接入、跨运营商通信和跨国通信等环境,对普通用户服务可达性、公司服务群及内网建设等有较大负面影响。为了规避此情况,在过去进行技术选型时,公网通信常常选用 TCP 协议,甚至是直接选用基于 TCP 协议的 HTTP 协议(i.g. HTTP/1.1、HTTP/2),以保证服务的可达性和质量。

\subsection{新旧上层协议}

在过去基于 TCP 的 HTTP 协议中,HTTP/1.1 和 HTTP/2 是主流选择。其中 HTTP/1.1 由于每次请求都会打开一个新连接,资源消耗大且交互延迟高;而 HTTP/2 则可以复用 TCP 连接,部分解决了新连接开销大的问题,但也存在队头阻塞的问题。队头阻塞是指 TCP 拥塞控制中多个 HTTP 连接复用同一个 TCP 连接时,假如 TCP 消息队列队头的包发生了超时丢包,则此丢包会阻塞后续所有的 HTTP 连接,即使这些 HTTP 连接之间没有依赖关系。<TOREF>尽管有 BBR\cite{45646} 等项目致力于在 TCP 发生超时丢包时优化拥塞控制策略、尽快恢复原传输速率,但仍然无法彻底解决这个问题。

HTTP/3 的提出便是为了解决这个问题。HTTP/3 由 Google 主导开发,其最初是在 UDP 协议上实现了能够进行稳定传输的 QUIC 协议,然后在 2018 年提出将 HTTP over QUIC 重命名为 HTTP/3<TOREF>,并于同年获得了 IETF 成员的认可<TOREF>。尽管截至目前 HTTP/3 仍是草案状态\cite{ietf-quic-http-34},但浏览器 Google Chrome 和 Firefox、CDN 厂商 Cloudflare 均已部署了对 HTTP/3 的支持。由于 HTTP/3 基于 QUIC 而 QUIC 基于 UDP,在 HTTP/3 这样的新一代上层协议中,UDP 流量质量不佳问题的影响面开始逐渐扩大。

不仅有 HTTP/3,WireGuard 也是一个例子。WireGuard 是 Linux 5.6 被合并入 Linux 内核<TOREF>的、新一代 VPN 协议。WireGuard 具有全平台、高性能、易配置、支持后量子安全<TOREF>等优点,能够极大地方便公司内网建设、自建 AS 等组网问题。WireGuard 选择了 UDP 协议来进行节点间数据传输,从而规避了 TCP over TCP 时的性能问题,并允许复用现有的 UDP over TCP 来降低目前实现的复杂度。此选择同样推动了对于解决 UDP 流量质量不佳问题方法的探索。

\subsection{UDP over TCP}

为了解决 UDP 流量质量不佳的问题,目前的主流方案是利用 TCP 流量的关键性,将 UDP 流量伪装为 TCP 流量,亦即 UDP over TCP。具体实现有 GOST<TOREF> 等项目。也有一些其他方案,例如:UDP over HTTP,从而不必自行管理 TCP 连接;UDP over WebSocket,从而提高实时性并方便兼容非 TCP 的流式协议。

但这些 UDP over TCP 或者是其他更上层协议的方案存在一些问题,例如:在调用操作系统接口时,协议的封装和解封装是在完整通过 Linux 内核网络栈后在用户态进行的,一方面通过 Linux 内核网络栈的过程和通用 Linux API 的限制使得应用存在性能优化空间,另一方面用户态获得的原始包无法直接复用 Linux 内核网络栈的各个工具(e.g. TC、Netfilter、IP Set 等)。

本项目依然在协议上选择 UDP over TCP,但是借助 Linux 可编程内核机制 eBPF,直接提前接管 L2-L4 的协议处理,从而一方面不必实现完整的下层 TCP 协议、仅需模拟 UDP 流量为 TCP 流量实现伪装,另一方面完成封装或解封装后 Linux 内核网络栈工具的复用。

\section{工具背景}

\subsection{eBPF}

eBPF 是 Linux 内核提供的一套编程机制。它允许开发者安全地扩展 Linux 内核功能而不需要修改内核代码、不需要重新编译内核。eBPF 以一个二进制文件的形式被加载到内核中,首先经由 BPF verifier 校验安全性,校验通过后被解释执行,其中 runtime 带有 JIT,并允许通过 maps 进行数据持久化并与用户态程序沟通<TOREF>。

早期 BPF 作为网络包处理机制,架构较简单,功能非常受限。但从 2011 年开始 BPF 机制收到了 Linux 内核开发者的进一步开发,从而使得其功能获得了大的增强。改进增强后的 BPF 则称为 eBPF(i.e. extended BPF)或是直接沿用 BPF 这个名称,而曾经的 BPF 则称作 cBPF(i.e. classic BPF)。\cite{10.1145/3371038}

\subsection{XDP 和 TC BPF}

XDP 和 TC BPF 是 eBPF 在网络包处理方面的接口。其中 XDP 仅工作在入方向,位于整个 Linux 内核网络栈最前端、进行于 SKB 分配之前。而 TC BPF 而可以工作在出入两方向,进行于 SKB 分配之后。本项目中我们选择 XDP 和 TC BPF 出方向这两个接口进行开发,从而在 eBPF 中实现透明的 UDP over TCP。
